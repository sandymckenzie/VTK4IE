There is a growing body of evidence that demonstrates measurable benefits
attained when exploring scientific data using immersive interfaces, such as
molecular research at UNC~\cite{Brooks:1990},
genetics at NCSA~\cite{Brady:1995},
oil well placement at the University of Colorado~\cite{Gruchalla:2004},  confocal microscopy data at Brown University~\cite{Prabhat:2008}, and interdisciplinary immersive visual analytics at the Electronic Visualization Laboratory~\cite{Marai:2016} to name but a few.
While it is not scientifically verifiable, any time scientists
express situations where they "discovered" some relationship in their data
while immersed in a virtual reality (VR) system, we can make the case that the
interface provided a utility that helped them advance their work.

Yet, knowing that there are benefits is only half the equation.
The other half is the cost.
A considerable contribution to the cost | one that is often not
formulated | is personnel time to get data into the VR system.
That time expense is often exacerbated due to a lack of tools that
allow data to be directly imbibed into a virtual environment.

A path that many research teams have taken is to use the established and
feature-rich Visualization Toolkit (VTK).
VTK is an application programming interface (API) that provides quick access to an expanse
of scientific visualization rendering algorithms, as well as to components
for displaying and interacting with the results on a desktop.
While the concept of combining VTK with VR is sound, the
compatibility of VTK with other rendering software presented a difficult
challenge.  
There were several reasonably successful attempts at this amalgamation,
but in the end, there were either too many inefficiencies to allow the
software to be adequately interactive, or the melding was too fragile to
maintain as VTK and the VR libraries each evolved.

Consequently, the better solution was to adapt VTK to enable it to be
more easily integrated into other rendering systems.
Thus, we adapted VTK by adding new options for rendering.
Rather than only being able to render into windows with graphics contexts
created by VTK itself, it is now the possible to "externally" render
into contexts provided by a collaborating system, or even to integrate a
VR system directly into VTK.

Immersive visualization efforts are often associated with research
facilities that provide large-scale VR systems such as CAVEs\texttrademark and
other large-screen walk-in displays.
There is also a growing audience of potential VR users who can now
gain access to immersive interfaces through the new abundance and low-cost
of head-mounted displays (HMDs).
Ideally, there would be one solution to reach both audiences. While this is
technically possible, the consumer systems offer a simpler
approach that will entice many developers to follow that path.
Thus, we offer two approaches, one that addresses the simpler solution
of integrating VTK directly with Oculus or OpenVR and another that allows the integration of VTK with
any full-fledged VR integration library that is capable of interfacing
with CAVE-style and HMD displays.

\textbf{OpenGL context sharing}.
Our \texttt{vtkRenderingExternal} VTK module provides a complete integration API,
including proper lighting, interaction, picking and access to the entire
VTK pipeline.
This, enables simple utilization for application developers using
any OpenGL-based VR Toolkit.

\textbf{VR Toolkit embedding}.
The Oculus and OpenVR VTK modules support several immersive environments directly
without the issues faced by previous work, and it is a complete template
for embedding other VR toolkits within VTK in future work.

\textbf{Enhanced performance}.
As the nature of immersive interfaces, especially HMDs, requires
high-performance rendering, our effort also includes VTK rendering enhancements
including the following:

\begin{compactitem}
\item a new default OpenGL 3.2+ pipeline;
\item dual depth peeling for transparency; and 
\item symmetric multiprocessing (SMP) tools and algorithms.
\end{compactitem}

In the sections that follow, we illustrate how our amalgamation of VTK and VR toolkits support our goals for enhancing scientific visualization through immersive environments.

